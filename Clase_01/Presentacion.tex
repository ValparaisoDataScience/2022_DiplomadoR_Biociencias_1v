% Options for packages loaded elsewhere
\PassOptionsToPackage{unicode}{hyperref}
\PassOptionsToPackage{hyphens}{url}
%
\documentclass[
  ignorenonframetext,
]{beamer}
\usepackage{pgfpages}
\setbeamertemplate{caption}[numbered]
\setbeamertemplate{caption label separator}{: }
\setbeamercolor{caption name}{fg=normal text.fg}
\beamertemplatenavigationsymbolsempty
% Prevent slide breaks in the middle of a paragraph
\widowpenalties 1 10000
\raggedbottom
\setbeamertemplate{part page}{
  \centering
  \begin{beamercolorbox}[sep=16pt,center]{part title}
    \usebeamerfont{part title}\insertpart\par
  \end{beamercolorbox}
}
\setbeamertemplate{section page}{
  \centering
  \begin{beamercolorbox}[sep=12pt,center]{part title}
    \usebeamerfont{section title}\insertsection\par
  \end{beamercolorbox}
}
\setbeamertemplate{subsection page}{
  \centering
  \begin{beamercolorbox}[sep=8pt,center]{part title}
    \usebeamerfont{subsection title}\insertsubsection\par
  \end{beamercolorbox}
}
\AtBeginPart{
  \frame{\partpage}
}
\AtBeginSection{
  \ifbibliography
  \else
    \frame{\sectionpage}
  \fi
}
\AtBeginSubsection{
  \frame{\subsectionpage}
}
\usepackage{lmodern}
\usepackage{amssymb,amsmath}
\usepackage{ifxetex,ifluatex}
\ifnum 0\ifxetex 1\fi\ifluatex 1\fi=0 % if pdftex
  \usepackage[T1]{fontenc}
  \usepackage[utf8]{inputenc}
  \usepackage{textcomp} % provide euro and other symbols
\else % if luatex or xetex
  \usepackage{unicode-math}
  \defaultfontfeatures{Scale=MatchLowercase}
  \defaultfontfeatures[\rmfamily]{Ligatures=TeX,Scale=1}
\fi
\usetheme[]{Malmoe}
\usecolortheme{beaver}
\usefonttheme{structurebold}
% Use upquote if available, for straight quotes in verbatim environments
\IfFileExists{upquote.sty}{\usepackage{upquote}}{}
\IfFileExists{microtype.sty}{% use microtype if available
  \usepackage[]{microtype}
  \UseMicrotypeSet[protrusion]{basicmath} % disable protrusion for tt fonts
}{}
\makeatletter
\@ifundefined{KOMAClassName}{% if non-KOMA class
  \IfFileExists{parskip.sty}{%
    \usepackage{parskip}
  }{% else
    \setlength{\parindent}{0pt}
    \setlength{\parskip}{6pt plus 2pt minus 1pt}}
}{% if KOMA class
  \KOMAoptions{parskip=half}}
\makeatother
\usepackage{xcolor}
\IfFileExists{xurl.sty}{\usepackage{xurl}}{} % add URL line breaks if available
\IfFileExists{bookmark.sty}{\usepackage{bookmark}}{\usepackage{hyperref}}
\hypersetup{
  pdftitle={CLASE 01 - PRESENTACIÓN DEL DIPLOMADO},
  pdfauthor={Dr.~José Gallardo Matus \textbar{} https://genomics.pucv.cl/},
  hidelinks,
  pdfcreator={LaTeX via pandoc}}
\urlstyle{same} % disable monospaced font for URLs
\newif\ifbibliography
\setlength{\emergencystretch}{3em} % prevent overfull lines
\providecommand{\tightlist}{%
  \setlength{\itemsep}{0pt}\setlength{\parskip}{0pt}}
\setcounter{secnumdepth}{-\maxdimen} % remove section numbering
\newcommand{\columnsbegin}{\begin{columns}}
\newcommand{\columnsend}{\end{columns}}

\title{CLASE 01 - PRESENTACIÓN DEL DIPLOMADO}
\subtitle{Diplomado en Análisis de Datos con R e Investigación reproducible para
Biociencias.}
\author{Dr.~José Gallardo Matus \textbar{} \url{https://genomics.pucv.cl/}}
\date{29 August 2022}
\institute{Pontificia Universidad Católica de Valparaíso}

\begin{document}
\frame{\titlepage}

\begin{frame}{PLAN DE LA CLASE INAUGURAL}
\protect\hypertarget{plan-de-la-clase-inaugural}{}

\begin{itemize}
\item
  Palabras de bienvenida.
\item
  Presentación de los participantes.
\item
  Revisión programa del Diplomado.
\item
  Lectura condiciones de operación AEA.
\item
  Habilitación recursos de comunicación y aprendizaje.
\end{itemize}

\end{frame}

\begin{frame}{PALABRAS DE BIENVENIDA}
\protect\hypertarget{palabras-de-bienvenida}{}

\textbf{Dr.~José Gallardo Matus}\\
Director del Diplomado.

\begin{itemize}
\tightlist
\item
  \textbf{Excelencia académica}\\
  PUCV Acreditada por 7 años, máximo a nivel nacional. Doctorado en
  Biotecnología acreditado por 5 años: 85 graduados.\\
  Vinculación permanente con la industria.\\
  Recursos de enseñanza y aprendizaje de nivel mundial.
\end{itemize}

\end{frame}

\begin{frame}{PROFESORES Y AYUDANTES DIPLOMADO}
\protect\hypertarget{profesores-y-ayudantes-diplomado}{}

\textbf{Dr.~José Gallardo}\\
Profesor de genética y genómica aplicada\\
Doctor en Ciencias\\
Profesor adjunto PUCV

\textbf{Dr.~María Angélica Rueda}\\
Profesora de modelamiento predictivo\\
Doctora en Ciencias Agropecuarias\\
Investigadora postdoctoral PUCV

\textbf{Dr(c) Jaqueline Flores}\\
Coordinadora postulación y matrículas.

\textbf{Mag. Paz Caballero}\\
Ayudante del diplomado.

\end{frame}

\begin{frame}{PRESENTACIÓN DE LOS PARTICIPANTES}
\protect\hypertarget{presentaciuxf3n-de-los-participantes}{}

\includegraphics[width=1\linewidth]{Welcome}

\end{frame}

\begin{frame}{OBJETIVOS DEL DIPLOMADO}
\protect\hypertarget{objetivos-del-diplomado}{}

Al finalizar el Diplomado los alumnos serán capaces de:

• Realizar análisis exploratorio de datos.\\
• Aplicar modelos básicos de inferencia estadística paramétrica y no
paramétrica.\\
• Seleccionar y aplicar modelos lineales, no lineales y multivariados de
modelamiento predictivo.\\
• Compartir y presentar resultados de análisis de datos de manera
moderna y profesional mediante reportes dinámicos generados con el
software Rmarkdown.\\
• Trabajar bajo el paradigma de la investigación reproducible utilizando
de forma autónoma el software de control de versiones GitHub.

\end{frame}

\begin{frame}{CONTENIDOS}
\protect\hypertarget{contenidos}{}

\begin{itemize}
\item
  \textbf{UNIDAD 1. Investigación reproducible y análisis exploratorio
  de datos.}\\
  \emph{Palabras clave: R, Rstudio, Rmarkdown, Github, variables
  aleatorias, distribución de probabilidad, análisis exploratorio de
  datos.}
\item
  \textbf{UNIDAD 2. Pruebas de contraste de hipótesis paramétricas y no
  paramétricas.}\\
  \emph{Palabras clave: Parámetro, estadístico, correlación,
  permutación, combinación, inferencia estadística, contraste de
  hipótesis.}
\item
  \textbf{UNIDAD 3. Modelos lineales, análisis de supervivencia y
  multivariante. } \emph{Palabras clave: Regresión lineal, regresión
  lineal múltiple, regresión logística, análisis de supervivencia,
  análisis de cluster, análisis de componentes principales.}
\end{itemize}

\end{frame}

\begin{frame}{EVALUACIÓN DEL DIPLOMADO}
\protect\hypertarget{evaluaciuxf3n-del-diplomado}{}

\begin{itemize}
\item
  La evaluación del diplomado consiste en el desarrollo de un proyecto
  personal de análisis de datos de acuicultura con R.
\item
  Se dará énfasis a que los alumnos resuelvan un problema usando datos
  de su propio trabajo o investigación.
\item
  El trabajo se desarrolla en dos etapas, la primera pondera un 40\% y
  la segunda un 60\% de la nota final.
\end{itemize}

\end{frame}

\begin{frame}{CONDICIONES DE APROBACIÓN DEL DIPLOMADO}
\protect\hypertarget{condiciones-de-aprobaciuxf3n-del-diplomado}{}

\begin{itemize}
\item
  \textbf{Nota mínima}: 4,0 en escala de 1-7 con 60\% de exigencia.
\item
  \textbf{Asistencia a clases sincrónicas}: 80\%. Esto es independiente
  de que las calificaciones parciales o totales sean mayores de 4,0.
\item
  \textbf{Plazo de entrega de reportes}: No entregar los reportes en los
  plazos establecidos para ello será calificado con la nota mínima
  (1,0).
\end{itemize}

\end{frame}

\begin{frame}{REQUISITOS}
\protect\hypertarget{requisitos}{}

\begin{itemize}
\item
  \textbf{Título}: Título profesional o licenciatura.
\item
  \textbf{Programación básica con R}: Deseable pero no es excluyente.\\
  Alumnos sin experiencia previa deben considerar 4 horas de estudio
  adicional por semana para alcanzar nivel avanzada de los objetivos de
  aprendizaje.
\item
  \textbf{Inglés}: Los softwares R, Rstudio, Rmarkdown, GitHub y todas
  las librerías de análisis estadístico que se usarán en el curso solo
  están disponibles en inglés. Alumnos sin competencias de lectura en
  inglés no deberían tomar el curso.
\end{itemize}

\end{frame}

\begin{frame}{CONDICIONES DE OPERACIÓN}
\protect\hypertarget{condiciones-de-operaciuxf3n}{}

\begin{itemize}
\item
  \textbf{Nombre de la actividad de extensión académica:}\\
  Diplomado Análisis de datos con R e investigación reproducible para
  Biociencias.
\item
  \textbf{Resolución:}\\
  105/2022.
\item
  \textbf{Fecha de ejecución:}\\
  INICIO: 29/08/2022 TÉRMINO: 30/10/2022
\item
  \textbf{Consultas, sugerencias y reclamos del curso:}
  \href{mailto:genomica.aplicada@pucv.cl}{\nolinkurl{genomica.aplicada@pucv.cl}}\\
  \href{mailto:oct@pucv.cl}{\nolinkurl{oct@pucv.cl}}
\end{itemize}

\end{frame}

\begin{frame}{RECURSOS DE APRENDIZAJE Y COMUNICACIÓN}
\protect\hypertarget{recursos-de-aprendizaje-y-comunicaciuxf3n}{}

\begin{itemize}
\item
  \textbf{Repositorio}: Todo el material disponible en Google Drive.
\item
  \textbf{Material docente}: Diapositivas de clases, videos, guías de
  aprendizaje y códigos de programación para el análisis de datos con R.
\item
  \textbf{SLACK}: Foro de comunicación y preguntas.
\item
  \textbf{R}: Acceso a versión open source en la nube.
\item
  \textbf{Rstudio}: Acceso a espacio de trabajo Rstudio.cloud por 4
  meses.
\item
  \textbf{Github}: Acceso a versión gratuita.
\end{itemize}

\end{frame}

\end{document}
